\documentclass[a4paper,10pt]{article} 
\usepackage[utf8]{inputenc}
%\usepackage{geometry}
\usepackage[a4paper, total={6in, 8in},top=1in]{geometry}
\usepackage{verbatim}
\usepackage{graphicx}

\title{Clockwork RNN}
\author{Paul Mustière - David Panou}

\begin{document}
\maketitle

\section{Abstract}

Quickly introduce the RNN Game

\section{Context}

Comparition with \textit{https://arxiv.org/pdf/1412.7753v2.pdf} and previous work

\section{Clockwork RNN}

\section{Article contribution}

\section{Paper's results and further analysis}

\section{Our benchmark}

How do we plain to evaluate our implementations

A quick recap of some article.

TODO: Change thebibliography to BibTex (Much faster to use)

\begin{comment}
\begin{thebibliography}{9}

\bibitem{turing_machines}Graves, Alex, Wayne, Greg, and Danihelka, Ivo. Neural turing machines. arXiv preprint arXiv:1410.5401, 2014.

\bibitem{memory_networks} J. Weston, S. Chopra, A. Bordes. Memory Networks. arXiv preprint arXiv:1410.5401, 2014.

\bibitem{turing_complete} Siegelmann, H. T. and Sontag, E. D. (1995). On the computational power of neural nets.
Journal of computer and system sciences, 50(1):132–150.

\bibitem{stack} S. Das, C.L. Giles, G.Z. Sun, "Learning Context Free Grammars: Limitations of a Recurrent Neural Network with an External Stack Memory," Proc. 14th Annual Conf. of the Cog. Sci. Soc., p. 79, 1992.

\bibitem{fast-weights} J. Schmidhuber. Learning to Control Fast-Weight Memories : An Alternative To Dynamic Recurrent Networks 
\textit{Neural Computation, 4(1):131-139, 1992}. 

\bibitem{self_learning} Hochreiter, Sepp; Younger, A. Steven; Conwell, Peter R. (2001). "Learning to Learn Using Gradient Descent". \textit{ICANN 2001, 2130: 87–94}

\bibitem{qa1} Berant, Jonathan, Chou, Andrew, Frostig, Roy, and Liang, Percy. Semantic parsing on freebase from question-answer pairs. \textit{In EMNLP, pp. 1533–1544, 2013.}

\bibitem{qa2} Berant, Jonathan, Srikumar, Vivek, Chen, Pei-Chun, Huang, Brad, Manning, Christopher D, Van- der Linden, Abby, Harding, Brittany, and Clark, Peter. Modeling biological processes for reading comprehension. \textit{In Proc. EMNLP, 2014.}

\bibitem{switchboard} Jurafsky, Daniel, Rebecca Bates, Noah Coccaro, Rachel Martin, Marie Meteer, Klaus Ries, Elizabeth Shriberg, Andreas Stolcke, Paul Taylor, and Carol Van Ess-Dykema. 1998. Switchboard Discourse Language Modeling Project Report Research Note 30, Center for Speech and Language Processing, Johns Hopkins University, Baltimore, MD

\bibitem{fractalnet} Gustav Larsson, Michael Maire, and Gregory Shakhnarovich. Fractalnet: Ultra-deep neural networks without residuals. arXiv preprint arXiv:1605.07648, 2016.

\bibitem{wavenet} van den Oord, A. et al. Wavenet: A generative model for raw audio. arXiv preprint arXiv:1609.03499 (2016). 1609.03499

\bibitem{quald} http://qald.sebastianwalter.org/

\bibitem{bordes} Bordes, Antoine, Weston, Jason, and Usunier, Nicolas. Open question answering with weakly su- pervised embedding models. ECML-PKDD, 2014b.

\bibitem{fader} Fader, Anthony, Zettlemoyer, Luke, and Etzioni, Oren. Paraphrase-driven learning for open question answering. In ACL, pp. 1608–1618, 2013.

\bibitem{sutskever} Sutskever, I. Vinyals, O., Le. Q. V. Sequence to sequence learning with neural networks. In Proc. Advances in Neural Information Processing Systems 27 3104–3112 (2014). 

\end{thebibliography}
\end{comment}

\end{document}
